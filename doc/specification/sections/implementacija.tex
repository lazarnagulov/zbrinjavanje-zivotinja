\lstset{language=C++,
        basicstyle=\ttfamily,
        frameshape={}{yy}{}{},
        breaklines=true,
        postbreak=\raisebox{0ex}[0ex][0ex]{\ensuremath{\color{gray}\hookrightarrow\space}},
        morekeywords={string, AnimalType, readonly, IReadOnlyCollection, Photo, List, where, interface, Account, var, SHA256, Encoding, UTF8, Convert,
        get, set, ref, Guid, Comment, RelayCommand, PostViewModel, REPLACE, FUNCTION, RETURNS, IF, RAISE, NEW, INotificationRepository, Notification, Person,
        CREATE, OR, TRIGGER, DECLARE, BEGIN, SELECT, INTEGER, BEFORE, INSERT, UPDATE, FOR, EACH, ROW, EXECUTE, LANGUAGE, INTO, FROM, ON, JOIN, WHERE, THEN, END, RETURN, AS, RETURNS, EXCEPTION},
        keywordstyle=\color{blue}\ttfamily,
        stringstyle=\color{red}\ttfamily,
        commentstyle=\color{green}\ttfamily,
        morecomment=[l][\color{magenta}]{\#},
        showstringspaces=false
}


\rhead{10. Implementacija}
\section{Implementacija}
\par Implementacija je javno dostupna na githubu: 
\begin{center}
    https://github.com/lazarnagulov/zbrinjavanje-zivotinja
\end{center}
\subsection{Primena MVVM šablona}
\par Za razvoj je korišćen \textbf{MVVM} (Model-View-ViewModel) arhitektonski obrazac. 
\subsubsection*{Model}
\par \textbf{Model} čine \textbf{entiteti}, \textbf{servisi} i \textbf{repozitorijumi} (domenski sloj). 
\subsubsection*{Entiteti}
\par \textbf{Entiteti} su strukture podataka koje, pored samih podataka, sadrže pomoćne funkcije za upravljanje sa konteinerima. 
\par Na primeru [Listing \ref{list:animal}], klasa \textit{Animal} sadrži listu fotografija i pomoćne funkcije za dodavanje, brisanje i dobavljanje
readonly liste.
\begin{lstlisting}[caption={Primer entiteta}, captionpos=b, label=list:animal]
public class Animal(AnimalType type, string name, int age, string description)
{
    public AnimalType Type { get; set; } = type;
    public string Name { get; set; } = name;
    public int Age { get; set; } = age;
    public string Description { get; set; } = description;
    private readonly List<Photo> _photos = [];
    public IReadOnlyCollection<Photo> Photos 
        => _photos;

    public void AddPhoto(Photo photo) 
        => _photos.Add(photo);
    public void RemovePhoto(Photo photo) 
        => _photos.Remove(photo);
}
\end{lstlisting}
\subsubsection*{Servisi}
\par \textbf{Servisi} služe za komunikaciju sa podacima i poslovnom logikom. U njima se nalazi komplikovanija poslovna logika koja može da uzima u obzir više objekata.
Koristi interfejse repozitorijuma za pristup podacima u bazi. 
\begin{lstlisting}[caption={Primer servisa}, captionpos=b, label=list:service]
public class NotificationService(INotificationRepository notificationRepository)
{
    public bool Insert(Notification notification) 
       => notificationRepository.Insert(notification);
    public bool Delete(Notification notification) 
       => notificationRepository.Delete(notification);
    public Notification? GetById(Guid id) 
       => notificationRepository.GetById(id);
    public List<Notification> GetAll() 
       => notificationRepository.GetAll();
    public bool SendNotification(Person recipient, string message)
    {
        var notification = new Notification(recipient, message);
        return Insert(notification);
    }
    public bool DeleteById(Guid id)
    {
        var notification = GetById(id);
        return notification is not null && notificationRepository.Delete(notification);
    }

    public List<Notification> GetAllIncluded()
        => notificationRepository.GetAllIncluded();

    public List<Notification> GetAllForPerson(Person person)
        => notificationRepository
            .GetAllIncluded()
            .Where(n => n.Recipient.Id == person.Id)
            .ToList();
}
\end{lstlisting}
\par Primer servisa se nalazi na listingu \ref{list:service}. On je zadužen za dobavljanje, kreiranje i brisanje svi obaveštenja, dobavljanje svih objekata
koji su u relaciji sa obaveštenjem, kao i dobavljanje svakog obaveštenja za određenu \textit{osobu}.
\subsubsection*{Repozitorijumi}
\par \textbf{Repozitorijumi} služe za pristup podacima. Najosnovnije operacije nad \textit{entitetima} u bazi podataka su \textbf{CRUD} (create, read, update, delete) 
operacije, koje se nalaze u \textit{ICrud$<$T$>$} intefejsu, prikazanom na listingu \ref{list:crud}.
\begin{lstlisting}[caption={ICrud$<$T$>$ interfejs}, captionpos=b, label={list:crud}]
public interface ICrud<T> 
    where T : class
{
    List<T> GetAll();
    T? GetById(Guid id);
    bool Insert(T entity);
    bool Delete(T entity);
    bool Update(T entity)
}
\end{lstlisting}
\subsubsection*{View}
\par \textbf{View} je korisnički interfejs aplikacije. Odgovoran je za prikaz podataka i interakcijom sa korisnikom,
ali ne sadrži poslovnu logiku. 
\par Obično se implementira pomoću \textbf{XAML}. Podatke dobavlja sa \textit{viewmodela} povezivanjem podataka (eng. data binding).
\par Aplikacije sadrži 5 glavnih prozora - za uloge u sistemu i početni. Sa početnog prozora se pristupa drugim prozorima prijavom na sistem. U svakom prozoru se smenjuju više \textit{view}-ova.
To se postiže \textbf{navigacijom}. 
\par Svaki \textit{view} je povezan sa jedniom \textit{viewmodel} klasom, koja sadrži kontekst - podatke koje se koriste na samom \textit{view}-u.
\subsubsection*{ViewModel}
\par \textbf{ViewModel} je posrednički sloj između korisničkog interfejsa i domenskog sloja.
\par Za svaki entitet je napravljen \textbf{omotač} (eng. wrapper) [Listing \ref{list:wrap}], kako bi se prilagodilo korisničkom interfejsu. Svaki od svojstava ovih klasa se može povezati
(eng. binding) za \textit{xmal} dokument koji predstavlja \textit{view}. 
\begin{lstlisting}[caption={Primer ViewModel omotača}, captionpos=b, label=list:wrap]
public class ViewModelBase : INotifyPropertyChanged 
{/*...*/}
public class PhotoViewModel(Photo photo) : ViewModelBase
{
    public Guid Id
    {
        get => _id;
        set => SetField(ref _id, value);
    }

    public string Description
    {
        get => _description;
        set => SetField(ref _description, value);
    }

    public string Url
    {
        get => _url;
        set => SetField(ref _url, value);
    }

    private Guid _id = photo.Id;
    private string _description = photo.Description;
    private string _url = photo.Url;
}
\end{lstlisting}
\par Ključan element u svemu ovome je  \textbf{INotifyPropertyChanged} interfejs jer 
omogućava obaveštavanje korisničkog interfejsa o promenama na \textit{viewmodel}-u - \textbf{posmatrač} (eng. Observer).
\subsubsection*{Komande}
\par Bitnu ulogu u odvajanju logike korisničkih akcija od korisničkog interfejsa igra \textbf{ICommand} interfejs. Ovim se povećava modularnost i testiranje u 
aplikaciji. Postoje dve implementacije komandi: \textbf{Relay} komande [Listing \ref{list:relay}] i \textbf{Klasne} komande.  
\begin{lstlisting}[caption={Primer \textit{relay} komande za dodavanje komentara}, captionpos=b, label=list:relay]
public PostListingViewModel(/* ... */)
{
    /* ... */
    LikePostCommand 
       = new RelayCommand<PostViewModel>(LikeCommand);
    AddCommentCommand 
       = new RelayCommand<PostViewModel>(CommentCommand, CanComment);
}

private static bool CanComment(PostViewModel arg) 
    => !string.IsNullOrEmpty(arg.NewComment);

private void CommentCommand(PostViewModel obj)
{
    var comment = new Comment(_authenticationStore.LoggedUser!, obj.NewComment);
    obj.Comments.Add(new CommentViewModel(comment));
    _postService.AddComment(obj.Id, comment);
}
\end{lstlisting}
\subsection{Perzistencija podataka}
\par Za perzistenciju podataka je korišćena PostgresSQL relaciona baza podataka. U nastavku je opisan način na koji smo kreirali bazu podataka (migracijom)
i kreirali validaciju. Na kraju je dat primer funkcionisanja manjeg dela sistema za perzistenciju podataka.
\subsubsection*{Migracija}
\par Migracija je urađena pomoću \textbf{Entity} radnog okvira komandama:
\begin{lstlisting}[caption={Kreiranje migracije}, captionpos=b]
dotnet ef migrations add Initial
dotnet ef database update
\end{lstlisting}
\par Alat automatski generiše C\# kod koji predstavlja promene u modelu i kako te promene treba primeniti na bazu podataka. Neke detalje je moguće podesi preko 
koda, poput \textit{unique} ograničenja, prikazanom na listingu \ref{list:unique} ili preko anotacija [Listing \ref{list:annot}]. 
\begin{lstlisting}[caption={Primer postavljanja unique ograničenja}, label=list:unique, captionpos=b]
modelBuilder.Entity<Account>()
    .HasIndex(account => account.Email)
    .IsUnique();

modelBuilder.Entity<Account>()
    .HasIndex(account => account.Username)
    .IsUnique();
\end{lstlisting}
\begin{lstlisting}[caption={Primer postavljanja anotacija}, label=list:annot, captionpos=b]
[Column("username")]
[Required]
[MaxLength(30)]
public string Username { get; set; }
\end{lstlisting}
\subsection*{Konekcija}
\par Da bi smo se uspešno povezali na bazu, moramo uneti parametre konekcije [Listing \ref{list:params}]. Parametri se učitavaju iz
\textbf{user-secrets} json datoteke, a pristupa im se pomoću klase \textit{Configuration}.
\begin{lstlisting}[caption={Parametri konekcije}, captionpos=b, label={list:params}]
public string Host { get; }
public int Port { get; }
public string Username { get; }
public string Password { get; }
public string Database { get; }
\end{lstlisting}
\par Primer \textit{user-secret}-a se nalazi na listingu \ref{list:user-secret}.
\begin{lstlisting}[caption={Primer \textit{user-secret}}, captionpos=b, label={list:user-secret}]
{
    "Database": {
        "Host": "localhost",
        "Port": 5432,
        "DatabaseName": "PetCenter",
        "Username": "postgres",
        "Password": "1234"
    }
}
\end{lstlisting}
\par Konekciju uspostavljamo \textbf{UseNpgsql} metodom:
\begin{lstlisting}[caption={Uspostavljanje konekcije}, captionpos=b]
optionsBuilder.UseNpgsql(databaseCredentials.ConnectionString);
\end{lstlisting}
\subsubsection*{Validacija}
\par Validacija na nivou baze podataka je pisana u \textbf{SQL}-u. Najčešće korišćen
vid validacije je \textbf{BEFORE INSERT/UPDATE TRIGGER} [Listing \ref{list:trig}], koji nam omogućava da, pre samog upisa podataka, proverimo njihovu validnost.
\begin{lstlisting}[caption={Primer validacije pomoću trigera}, captionpos=b, label=list:trig]
CREATE OR REPLACE FUNCTION fn_validate_request()
RETURNS TRIGGER AS $$
DECLARE
	account_type INTEGER;
BEGIN
	SELECT acc_type
	INTO account_type
	FROM request r
	JOIN account a ON r.person_req_author = a.person_id_person
	WHERE r.person_req_author = NEW.person_req_author;

	IF account_type <> 0 THEN
		RAISE EXCEPTION 'Invalid account type for author, expected 0 but got: %', account_type;
	END IF;

	RETURN NEW;
END;
$$ LANGUAGE plpgsql;

CREATE OR REPLACE TRIGGER trg_validate_request
BEFORE INSERT OR UPDATE ON request
FOR EACH ROW
EXECUTE FUNCTION fn_validate_request();
\end{lstlisting}
\par Na listingu \ref{list:trig} proveravamo da li je kreator zahteva \textit{član} (vrednost 0).
\subsection{Zaštita šifre}
\par Za zaštitu šifre koristimo \textbf{SHA256} [Listing \ref{list:encript}]. 
\begin{lstlisting}[caption={Primer funkcije za zaštitu šifre}, captionpos=b, label=list:encript]
public static string Encode(string password)
{
    var sha = SHA256.Create();
    var asBytes = Encoding.UTF8.GetBytes(password);
    var hashed = sha.ComputeHash(asBytes);

    return Convert.ToBase64String(hashed);
}
\end{lstlisting}
\subsection{Organizacija foldera}
\par Za organizaciju foldera je korišćen \textbf{package by layer} obrazac. Koristimo 6 glavnih foldera:
\begin{enumerate}
    \item \textbf{Core} - aplikativni sloj, sadrži servise, klase koje čuvaju stanja sistema (Stores) i pomoćne klase.
    \item \textbf{Domain} - domenski sloj, sarži enumeracije, modele i interfejse za repozitorijum.
    \item \textbf{HostBuilder} - zadužen za \textbf{dependency injection}.
    \item \textbf{Migrations} - sadrži migracije za bazu podataka.
    \item \textbf{Repository} - sadrži sve repozitorijume i \textbf{data context}.
    \item \textbf{WPF} - sadrži kod za korisnički interfejs -  \textbf{view} i \textbf{viewmodel}.
\end{enumerate}