\lstset{language=C++,
        basicstyle=\ttfamily,
        frameshape={}{yy}{}{},
        breaklines=true,
        postbreak=\raisebox{0ex}[0ex][0ex]{\ensuremath{\color{gray}\hookrightarrow\space}},
        morekeywords={string, AnimalType, readonly, IReadOnlyCollection, Photo, List},
        keywordstyle=\color{blue}\ttfamily,
        stringstyle=\color{red}\ttfamily,
        commentstyle=\color{green}\ttfamily,
        morecomment=[l][\color{magenta}]{\#}
}
\rhead{10. Implementacija}
\section{Implementacija}
\par Implementacija je javno dostupna na githubu: 
\begin{center}
    https://github.com/lazarnagulov/zbrinjavanje-zivotinja
\end{center}
\subsection{Primena MVVM šablona}
\par Za razvoj je korišćen \textbf{MVVM} (Model-View-ViewModel) arhitektonski obrazac. 
\subsubsection*{Model}
\par \textbf{Model} čine \textbf{entiteti}, \textbf{servisi} i \textbf{repozitorijumi}. 
\par Entiteti su strukture podataka koje, pored samih podataka, sadrže pomoćne funkcije za upravljanje sa konteinerima. 
\par Na primeru [Listing \ref{list:animal}], klasa \textit{Animal} sadrži listu fotografija i pomoćne funkcije za dodavanje, brisanje i dobavljanje
readonly liste.
\begin{lstlisting}[caption={Primer entiteta}, captionpos=b, label=list:animal]
public class Animal(AnimalType type, string name, int age, string description)
{
    public AnimalType Type { get; set; } = type;
    public string Name { get; set; } = name;
    public int Age { get; set; } = age;
    public string Description { get; set; } = description;
    private readonly List<Photo> _photos = [];
    public IReadOnlyCollection<Photo> Photos 
        => _photos;

    public void AddPhoto(Photo photo) 
        => _photos.Add(photo);
    public void RemovePhoto(Photo photo) 
        => _photos.Remove(photo);
}
\end{lstlisting}
\subsection{Orgnizacija foldera}
\subsection{Perzistencija podataka}