\rhead{Funkcionalni zahtevi}
\section{Funkcionalni zahtevi}
\par Aplikacija za praćenje zbrinjavanja životinja omogućava olakšano pronalaženje novog vlasnika za životinje. 
\par Na osnovu dijagrama slučajeva korišćenja [Slika \ref{fig:use-case}] primećujemo da postoje četiri uloge u aplikaciji: neregistrovani korisnik (gost), član, volonter i administrator. 
U nastavku se nalazi opis funkcionalnosti za svaku od uloga.
\subsection{Korisnici aplikacije}
\begin{enumerate}
    \item \textbf{Neregistrovani korisnik (gost)}: mogućnost registracije unošenjem imena, prezimena, adrese (ulica, broj, grad, država), korisničkog imena i šifre;
    prijavljivanja na sistem unosom korisničkog imena i šifre; pregleda odobrenih objava.
    \item \textbf{Član}: poseduje sve mogućnosti kao i \textit{neregistrovani korisnik (gost)}. Dodatno može da kreira objavu (koja mora biti prihvaćena od strane
    \textit{volontera}) birajući već postojeći tip životinje, komentariše i lajkuje odobrene objave. Za svaku odobrenu objavu, ima mogućnost da pošalje zahtev za udomljavanje ili 
    privremeni smeštaj. On uvek ima uvid u sve svoje zahteve, gde po mogućnosti može da odustane od njih. Ukoliko se zahtev odobri od strane \textit{volontera}, stanje objave se menja, 
    ali i dalje se prikazuje korisnicima. U svakom trenutku nakon udomljavanja ili obezbeđenog privremenog smeštaja, \textit{član} može da oceni životinju sa brojem i komentarom, 
    kao i da je vrati nazad, ukoliko nije zadovoljan. članu se takođe nudi mogućnost doniranja novca za pomoć udruženju.
    \item \textbf{Volonter}: poseduje sve mogućnosti kao i \textit{član}, stim da je njegova objava automatski prihvaćena. Da bi \textit{član} postao \textit{volonter}, on mora
    da bude izglasan od strane drugih \textit{volontera} (prvog volontera dodaje \textit{administrator}). On je zadužen za prihvatanje objava od strane člana i sakrivanje već 
    prihvaćenih objava. Ima pristup listi svih ponuda o udomljavanju i obezbeđivanju privremenog smeštaja, gde može da prihvati ili odbije ponude. Dodatno je zadužen za unos uplate od 
    strane člana i raspoređivanje sredstava po životinjama. Takođe može da dodaje, briše i menja podatke o vrsti životinja. 
    \item \textbf{Administrator}: poseduje sve mogućnosti kao i \textit{član}. Dodatno dodaje, briše i menja informacije o udruženju i dodaje prvog \textit{volontera}.
\end{enumerate}